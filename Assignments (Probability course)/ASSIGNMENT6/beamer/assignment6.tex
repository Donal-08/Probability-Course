\let\negmedspace\undefined
\let\negthickspace\undefined
%\RequirePackage{amsmath}
\documentclass{beamer}
 \usepackage[utf8]{inputenc}
 \usepackage{graphicx}
 \usepackage{amsmath}
 \usepackage{mathrsfs}
\usepackage{txfonts}
\usepackage{stfloats}
\usepackage{bm}
\usepackage{cite}
\usepackage{cases}
\usepackage{subfig}
 \usepackage{amsfonts}
 \usepackage{amssymb}
 \usepackage{enumitem}
\usepackage{mathtools}
\usepackage{tikz}
\usepackage{circuitikz}
\usepackage{verbatim}
% \usepackage[breaklinks=false,hidelinks]{hyperref}
% \usepackage{listings}
% \usepackage{calc}
% \usepackage{float}
% \usepackage{longtable}
% \usepackage{multirow}
% \usepackage{multicol}
% \usepackage{color}
% \usepackage{array}
% \usepackage{hhline}
% \usepackage{ifthen}
% \usepackage{chngcntr}
\usetheme{CambridgeUS}
% \newcommand{\BEQA}{\begin{eqnarray}}
% \newcommand{\EEQA}{\end{eqnarray}}
% \newcommand{\define}{\stackrel{\triangle}{=}}
% \bibliographystyle{IEEEtran}
%\bibliographystyle{ieeetr}
% \def\inputGnumericTable{}
% \let\vec\mathbf
% \providecommand{\ex}[1]{\ensuremath{\E\left(#1\right)}}
 \providecommand{\pr}[1]{\ensuremath{\Pr\left(#1\right)}}
 \providecommand{\sbrak}[1]{\ensuremath{{}\left[#1\right]}}
% \providecommand{\lsbrak}[1]{\ensuremath{{}\left[#1\right.}}
% \providecommand{\rsbrak}[1]{\ensuremath{{}\left.#1\right]}}
 \providecommand{\brak}[1]{\ensuremath{\left(#1\right)}}
% \providecommand{\lbrak}[1]{\ensuremath{\left(#1\right.}}
% \providecommand{\rbrak}[1]{\ensuremath{\left.#1\right)}}
 \providecommand{\cbrak}[1]{\ensuremath{\left\{#1\right\}}}
% \providecommand{\lcbrak}[1]{\ensuremath{\left\{#1\right.}}
% \providecommand{\rcbrak}[1]{\ensuremath{\left.#1\right\}}}
% \providecommand{\res}[1]{\Res\displaylimits_{#1}}
% \newcommand{\myvec}[1]{\ensuremath{\begin{pmatrix}#1\end{pmatrix}}}
% \newcommand{\mydet}[1]{\ensuremath{\begin{vmatrix}#1\end{vmatrix}}}
% \newcommand{\question}{\noindent \textbf{Question: }}
% \newcommand{\solution}{\noindent \textbf{Solution: }}
% Title page details: 
% \vspace{3cm}
\title{Assignment 6 CBSE class 12} 
\author{Donal Loitam(AI21BTECH11009)}
\date{\today}
\logo{\large \LaTeX{}}     % LOGO

% \maketitle
% \newpage
% \bigskip
% \renewcommand{\thefigure}{\theenumi}
% \renewcommand{\thetable}{\theenumi}
\begin{document}
        \begin{frame}
            \titlepage 
          \end{frame}
          \logo{}
        
        \begin{frame}{Outline}     %%%%%%%%%%%%%%%%%%%%%%%%%%%%%%%%%%%%%%5
              \tableofcontents
            \end{frame}
        
    \section{Question}
        \begin{frame}{Question}
        Find the variance of the number obtained on a throw of an unbiased die.
        \end{frame}
    \section{Solution} 
        \begin{frame}{Solution}  
         The sample space of the die experiment is $S=\cbrak{1,2,3,4,5,6}$.Let $X$ denote the number obtained on the throw. Then $X$ is a random variable
        which can take values 1, 2, 3, 4, 5, or 6.   \\
        $\therefore \pr{X = 1} = \pr{X = 2} = \pr{X = 3} = \pr{X = 4} =\pr{X = 5} = \pr{X = 6} = \frac{1}{6}$ \\\\
        Therefore the probability distribution of $X$ is:-
        \begin{table}[ht!]
        \begin{center}
		\providecommand{\pr}[1]{\ensuremath{\Pr\left(#1\right)}}
\normalsize \begin{tabular}{|c|c|c|c|c|c|c|}

\hline
\textbf{X} & $1$ & $2$ & $3$ & $4$ & $5$ & $6$ \\
\hline
\textbf{\pr{X}} & $\frac{1}{6}$ & $\frac{1}{6}$ & $\frac{1}{6}$ & $\frac{1}{6}$ & $\frac{1}{6}$ & $\frac{1}{6}$  \\
\hline
\end{tabular}

		\vspace*{5pt}
		\caption{}
		\label{table:table}
        \end{center}	
	    \end{table}
	    \end{frame}
        
        \begin{frame}
        Now, $E(X)$ = Expected value 
         \begin{align}
         E(X) &=  \sum_{i=1}^{n} x_{i}p(x_{i})\\
         Here , E(X) &=  \sum_{i=1}^{6} i  \brak{{\frac{1}{6}}}\\
                     &= \frac{1}{6} \times \brak{\sum_{i=1}^{6} i}\\
                     &= \frac{1}{6} \times \frac{6(6+1)}{2}\\
                     &= \frac{1}{6} \times \frac{42}{2}\\
                     &= \brak{\frac{21}{6}} 
         \end{align}
        \end{frame}
        
        \begin{frame}
            \begin{align}
             E(X^{2}) &=  \sum_{i=1}^{n} x_{i}^{2}p(x_{i})=\sum_{i=1}^{6} i^{2}  \brak{{\frac{1}{6}}}\\
                      &= \frac{1}{6} \times \brak{\sum_{i=1}^{6} i^{2}}\\
                      &= \frac{1}{6} \times \frac{6(6+1)(12+1)}{6}\\
                      &= \brak{\frac{91}{6}}
             \end{align}
             \begin{align}
             \therefore Var(X) &= E(X^{2})-(E(X))^{2}\\
                          &=\frac{91}{6}-\brak{\frac{21}{6}}^{2}\\
                          &=\frac{91}{6} - \frac{441}{36} = \brak{\frac{35}{12}} 
            \end{align}
        \end{frame}
\end{document}
