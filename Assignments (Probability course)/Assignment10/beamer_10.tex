%%%%%%%%%%%%%%%%%%%%%%%%%%%%%%%%%%%%%%%%%%%%%%%%%%%%%%%%%%%%%%%
%
% Welcome to Overleaf --- just edit your LaTeX on the left,
% and we'll compile it for you on the right. If you open the
% 'Share' menu, you can invite other users to edit at the same
% time. See www.overleaf.com/learn for more info. Enjoy!
%
%%%%%%%%%%%%%%%%%%%%%%%%%%%%%%%%%%%%%%%%%%%%%%%%%%%%%%%%%%%%%%%


% Inbuilt themes in beamer
\documentclass{beamer}

% Theme choice:
\usetheme{CambridgeUS}

 \providecommand{\pr}[1]{\ensuremath{\Pr\left(#1\right)}}
 \providecommand{\sbrak}[1]{\ensuremath{{}\left[#1\right]}}
 \providecommand{\brak}[1]{\ensuremath{\left(#1\right)}}
 \providecommand{\cbrak}[1]{\ensuremath{\left\{#1\right\}}}

% Title page details: 
\title{Assignment 10} 
\author{Donal Loitam - AI21BTECH11009}
\date{\today}
\logo{\large \LaTeX{}}

\usepackage{hyperref}
\usepackage{mathtools}
\usepackage{amssymb}
\usepackage{amsmath}


\begin{document}

% Title page frame
\begin{frame}
    \titlepage 
\end{frame}

% Remove logo from the next slides
\logo{}


% Outline frame
\begin{frame}{Papoulis Ex 6.2}
TABLE OF CONTENTS
    \tableofcontents
\end{frame}


% Lists frame
\section{Question}
\begin{frame}{Problem}
\textbf{Ex 6.2:}
X and Y are independent and uniform in the interval \brak{0,a}. Find the p.d.f of :- \\(a)\;\;\;\;$ \frac{X}{Y}$ \\
   (b) $\frac{Y}{(X+Y)}$ \\
   (c) \small{$\left|X-Y\right|$} \\
   \end{frame}

\section{Solution}
\begin{frame}{Solution - Part(a)}
  $f_{XY}(x,y) = f_X(x)f_Y(y) = \dfrac{1}{a^2}$, \;\;\;\;\;\;   $0 < x \le a$, \;\; $0 < y \le a$\\
  (a)  F_Z(z) = \pr{\tfrac{X}{Y} \le z} = \pr{X \le zY}\\ 
  (i) \;  $ z < 1$ \\
 \begin{align}
  F_Z(z) &= \pr{X \le zY}\\
         &=  \int_{0}^{a} \int_{0}^{zy} \frac{1}{a}.\frac{1}{a} \,dx \,dy = \dfrac{z}{2} ,\;\;\; z \le 1
\end{align}
  
   (i) \;  $ z < 1$ \;\;\;
 \begin{align}
  F_Z(z) &= \pr{X \le zY}\\
         &= 1 - \int_{0}^{a} \int_{0}^{x/z} \frac{1}{a}.\frac{1}{a} \,dy \,dx \\
         &= 1 - \int_{0}^{1} \frac{x}{2} \,dx = 1 - \dfrac{1}{2z} ,\;\;\; z > 1
\end{align}
\end{frame}

\begin{frame}{Solution - Part(a)(Contd.)}
(a) \\
\;\;\;\;\;\;\;\;\;\;\;\;\;Hence, $f_Z(z)$ can be written as :-
        \[
        f_Z(z)=\left\{
                \begin{array}{ll}
                  \dfrac{1}{2}  ,\;\;\;\;\;\;\;\;\;\;\;\;\;\;\;\;   z \le 1 \\
                  \dfrac{1}{2z^2} , \;\;\;\;\;\;\;\;\;\;\;  \; \; z > 1  \\
                  \end{array}
              \right.
       \]
\end{frame}


\begin{frame}{Solution - Part(b)}

  (b) \\ 
 \begin{align}
 F_Z(z) &= \pr{Z \le z} = \pr{\tfrac{Y}{X+Y} \le z}\\ 
       &= \pr{\tfrac{X}{Y} \ge \tfrac{1}{z} - 1} = 1 - \pr{\tfrac{X}{Y} \le \tfrac{1-z}{z}}\\
         &= \left\{
                \begin{array}{ll}
                  \frac{1}{2} \brak{\tfrac{z}{1-z}}  ,\;\;\;\;\;\;\;\;\;\;\;\;\;\;\;\;   0 < z \le 1/2 \\
                  1-\tfrac{1}{2} \brak{\tfrac{1-z}{z}} , \;\;\;\;\;\;\;\;\;\;\;  \;  1/2 < z < 1  \\
                  \end{array}
              \right.
\end{align}
\[
        f_Z(z)=\left\{
                \begin{array}{ll}
                  \dfrac{1}{2(1-z)^2}  ,\;\;\;\;\;\;\;\;\;\;   0 < z \le 1/2 \\
                 \;\;\; \dfrac{1}{2z^2} , \;\;\;\;\;\;\;\;\;\;\;  \; \;\;\;\; 1/2 < z < 1  \\
                  \end{array}
              \right.
       \]
\end{frame}
\begin{frame}{Solution - Part(c)}

  (c) \\ 
 \begin{align}
 F_Z(z) &= \pr{Z \le z} = \pr{|X-Y| \le z}\\ 
       &= \pr{\cbrak{|X-Y| \le z}\cbrak{X \ge Y}} + \pr{\cbrak{|X-Y| \le z}\cbrak{X < Y}}\\
       &= \pr{X-Y \le z,X \ge Y} + \pr{Y-X \le z,X < Y}\\   
       &= \int_{0}^{\infty} \int_{y}^{y+z} f_{XY}(x,y) \,dx \,dy + \int_{0}^{\infty} \int_{x}^{x+z} f_{XY}(x,y) \,dy \,dx \\
        &= \int_{0}^{\infty} \int_{y}^{y+z} f_{XY}(x,y) \,dx \,dy + \int_{0}^{\infty} \int_{y}^{y+z} f_{XY}(y,x) \,dx \,dy \\
         &= \int_{0}^{\infty} \int_{y}^{y+z} \cbrak{f_{XY}(x,y) + f_{XY}(y,x) } \,dx \,dy
\end{align}
\end{frame}

\begin{frame}{Solution - Part(c)(Contd.)}
  In general, \\
  \begin{align}
      f_Z(z) &= \int_{0}^{\infty} \frac{d}{dz} \int_{y}^{y+z}f_{XY}(x,y)+ f_{XY}(y,x)\,dx \,dy \\
      &= \int_{0}^{\infty} \cbrak{f_{XY}(y+z,y)+ f_{XY}(y,y+z) } \,dy 
      \end{align}
Here, \;\;\;\;\;\;\;\;\;\;\;\;\;X \sim U(0,a), \;\;\;\;\; Y  \sim U(0,a) \\
\begin{align}
F_Z(z) &= 1 - \frac{1}{a_2}.2.\frac{(a-z)^2}{2} = 1 - \brak{1 - \frac{z}{a}}^2 \\
f_Z(z) &= \frac{2}{a}\brak{1 - \frac{z}{a}} \;\;\;\;\;\;\; 0 \le z \le a
\end{align}
    \end{frame}


% % Blocks frame
\section{Codes}
\begin{frame}{CODES}
 \begin{block}{Beamer}
         Download Beamer code from - \href{https://github.com/Donal-08/Assignment10/blob/main/beamer_10.tex}{Beamer}
    \end{block}
\end{frame} 

\end{document}
